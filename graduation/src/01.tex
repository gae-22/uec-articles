\chapter{序論}\label{chap: 序論}
\pagenumbering{arabic}
第\ref{chap: 序論}章では,本論文の研究背景となっているIoTおよびLPWA通信技術について説明し,研究目的を記す.

\section{研究背景}
    近年,自動運転や高度運転支援システムに代表されるV2X\footnote{Vehicle-to-Everything}通信技術,ならびに物流やインフラ点検などでの活用が期待されるドローン技術が急速な進展を見せている.
    これらの高度なアプリケーションを実現するためには,車両やドローンが高精細な地図データや周辺のセンサー情報,高画質映像などをリアルタイムで交換する必要があり,通信には従来を大幅に上回る超高速・大容量性が求められる.

    この要求に応える有力な技術として,ギガビット級の通信を可能にするミリ波周波数帯の活用が期待されている.
    しかし,ミリ波は周波数が高いことに起因する物理的特性から,建物や他の車両,樹木といった障害物による影響を受けやすく,電波が遮蔽されることで通信が容易に途絶してしまうという重大な欠点を持つ.
    移動体である車両やドローンは,走行・飛行環境が動的に変化するため,この遮蔽による通信の瞬断が頻発し,通信品質の安定性確保が極めて重要な課題となっている.

    このミリ波の不安定性を補うため,遮蔽に強く通信が安定している低い周波数帯(Sub-6GHz帯など)を連携させるマルチバンド通信が検討されている.
    しかし,単に複数の周波数帯を併用するだけでは,通信断が発生してからバンドを切り替えるリアクティブな対応となり,遅延や通信のオーバーヘッドが大きい.
    したがって,研究上の課題はミリ波の持つ大容量という利点を最大限に活かしつつ,その不安定さを低い周波数帯でシームレスに補うかという,インテリジェントなバンド連携手法を確立することにある.


\section{研究目的}
    本研究ではこの課題に対し,本研究では特にV2Vが持つ移動軌跡の予測がある程度容易であるという特性に着目する.
    この予測可能性を活用し,通信環境の悪化を事前に予測することで,よりプロアクティブ(先行的)な通信制御を目指す.
    具体的には,以下の手法を検討し,通信の安定化を図る.
    \begin{itemize}
        \item 一定以上の遮蔽が予測される場合,通信が途絶する前にミリ波から低周波数帯へ能動的にバンド切替を行う手法.
        \item 予期せぬ瞬断など,速やかな回復が期待できる状況では,ミリ波通信のエラーパケットのみを低周波数帯で再送し,通信を効率的に継続する手法.
    \end{itemize}
    これらの適応的なマルチバンド連携手法を構築することで,V2Vにおけるスループットと安定性の両立を目指すものである.\cite{8469875}
