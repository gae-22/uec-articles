\renewcommand{\bibname}{参考文献}
\newcommand{\acknowledgmentname}{謝辞}
\renewcommand{\appendixname}{付録}
\newcommand{\argmax}{\operatornamewithlimits{\mathrm{arg\,max}}}
\newcommand{\argmin}{\operatornamewithlimits{\mathrm{arg\,min}}}

% 適宜追加
\usepackage{here}
\usepackage{ascmac}
\usepackage{comment}
\usepackage{color}
\usepackage{graphicx}
\usepackage{multirow}
\usepackage{sotsuron}
\usepackage{pifont}
\usepackage{amsmath,cases}
\usepackage{amssymb}
\usepackage[]{multicol}
\usepackage{bm}
\usepackage{url}
\usepackage{cite}
\usepackage{color}
\usepackage[colorlinks=true,linkcolor=black,citecolor=black,urlcolor=black]{hyperref}
\usepackage{doi}

% DOIリンクの設定
\renewcommand{\doitext}{DOI: }
\urlstyle{same}

% 参照
% http://tug.ctan.org/tex-archive/macros/latex/contrib/cleveref/cleveref.pdf
\usepackage{cleveref}
\crefname{enumi}{}{}
\crefname{equation}{式}{式}
\crefname{figure}{図}{図}
\crefname{table}{表}{表}
\crefname{algorithm}{アルゴリズム}{アルゴリズム}
\crefformat{chapter}{第#2#1#3章}
\crefformat{section}{#2#1#3節}
\crefformat{subsection}{#2#1#3項}
\newcommand{\crefpairconjunction}{と}
\newcommand{\crefrangeconjunction}{から}
\newcommand{\crefmiddleconjunction}{,}
\newcommand{\creflastconjunction}{,および}

% 適宜追加
\graphicspath{
    {./figure/01/}
        {./figure/02/}
}

\bindermode% バインダ用余白設定

% 日本語情報
\jclass{卒業論文}
\jtitle{V2V向けマルチバンド連携による通信安定化手法の検討}
\jdepart{I\hspace{-.1em}I類}{情報通信工学プログラム}
\jauthor{鐘ヶ江 僚太}
\jnumber{2210177}
\jadvisor{藤井 威生}{教授}
\jhyear{7}
% 提出日
\jdeadhyear{8}
\jmonth{1}
\jday{10}

%------------------------------------------------------------------------%

\renewcommand{\baselinestretch}{1.05}
